\documentclass[a4paper, 10pt]{report}


\usepackage[utf8]{inputenc}
\usepackage{graphicx}
\usepackage{german}
\usepackage{mathtools}
\usepackage{setspace}
\usepackage{fancyhdr}
\usepackage{nameref}
\usepackage[hidelinks]{hyperref}
\usepackage{xcolor}
\usepackage[gen]{eurosym}
\usepackage{float}

\setcounter{tocdepth}{3}
\setcounter{secnumdepth}{3}
\renewcommand{\chaptername}{}

\pagestyle{fancy}
\renewcommand{\headrulewidth}{0.4pt}
\renewcommand{\footrulewidth}{0.4pt}
\fancyhf{}
\lhead{{\footnotesize \textnormal{325.040}}}
\rhead{{\footnotesize \textnormal{Projekt 47}}}
\cfoot{{\footnotesize \textnormal{\thepage}}}
%\lfoot{{\footnotesize \textnormal{xxx}}}
%\rfoot{{\footnotesize \textnormal{\today}}}


\hypersetup{
    colorlinks,
    linkcolor={black},
    citecolor={blue!50!black},
    urlcolor={blue!80!black}
   }




%----------------------------------------------------------------------------------------
%	TITLE PAGE
%----------------------------------------------------------------------------------------

\newcommand*{\titleGM}{\begingroup % Create the command for including the title page in the document
\hbox{ % Horizontal box
\hspace*{0.2\textwidth} % Whitespace to the left of the title page
\rule{1pt}{\textheight} % Vertical line
\hspace*{0.05\textwidth} % Whitespace between the vertical line and title page text
\parbox[b]{0.75\textwidth}{ % Paragraph box which restricts text to less than the width of the page

{\noindent\Huge\bfseries Kontinuierliche Simulation}\\[2\baselineskip] % Title
{\large 325.040 - Projekt 47 - \textit{Sommersemester 2016}}\\[4\baselineskip] % Tagline or further description
{\textsc{Fabian Wedenik - 1426866 \newline
	Alexander Wimmer - 1328958 \newline
	Felix Hochwallner - 1328839 \newline
	Oskar Fürnhammer - 1329133}} % Author name
\\ \\ \\ \\
{Studienkennzahl 033 282}

\vspace{0.5\textheight} % Whitespace between the title block and the publisher
{\includegraphics[width=0.5cm]{TU-Logo}\noindent \ \ E325 Institut für Mechanik und Mechatronik}\\[\baselineskip] % Publisher and logo
}}
\endgroup}

%----------------------------------------------------------------------------------------
%	BLANK DOCUMENT
%----------------------------------------------------------------------------------------

\begin{document}
\thispagestyle{empty} % Removes page numbers
\titleGM % This command includes the title page
\newpage

\tableofcontents %inhaltsverzeichnis

\listoffigures %abbildungsverzeichnis



%---------------------------------------------------------------------------------------------------Vorwort------------------------------------------------------------------------------------------------
\renewcommand{\thechapter}{} %um ziffern vor überschriften zu unterdrücken
\chapter{Vorwort}
\renewcommand{\thechapter}{1}
%
Sehr geehrte Damen und Herren, liebe Leser und Leserinnen!
\\
\\
Das vorliegende Protkoll wurde im Rahmen der ...

%-----------------------------------------------------------------------------------------------------------------------------------------------------------------------------------------------------------
%												Chapter ****
%-----------------------------------------------------------------------------------------------------------------------------------------------------------------------------------------------------------
\renewcommand{\thechapter}{}
\chapter{Aufgabenstellung}
VORERST NUR REINKOPIERT
\\
\\Beschreibung
Sowohl mit MATLAB als auch MapleSim soll ein mechanisches Modell eines geregelten Doppelpendels realisiert werden. Dabei soll unter anderem ein Vergleich zwischen klassischer textueller Programmierung in MATLAB und grafischer, blockorientierter Modellierung in MapleSim gezogen werden.
\\
\\Aufgabenstellung
\\Implementieren Sie das Modell mit MATLAB. Führen Sie einen Simulationslauf mit den angegebenen Parametern durch, plotten Sie die Auslenkung x sowie die beiden Winkel Phi1 und Phi2 über der Zeit und interpretieren Sie die Ergebnisse. Berechnen Sie mit MATLAB auch die Eigenwerte. Ist das System stabil? Begründen Sie Ihre Aussage.
\\
\\Bauen Sie das Modell mit MapleSim auf, testen Sie das Modell mit den angegebenen Parametern und vergleichen Sie die Ergebnisse mit jenen aus der MATLAB-Simulation.

\renewcommand{\thechapter}{}
\chapter{Modell}

%Modellbild
\begin{figure}[h]
\centering  %Zentrierung
{\includegraphics[width=15cm]{Modell_Doppelpendel}}
\caption{Mechanisches Modell eines stehenden Doppelpendels}
\end{figure}


\renewcommand{\thechapter}{2}
%
%%%%%%%%%%%%%%%%%%%%%%%%%%%%%%%%%%%%%%%%%%%%%%%%% *titel von section eventuell subsection wegen uebersicht*
%
\section{foo}
xxx \\ xxx \\ xxx \\ xxx \\ xxx \\ xxx \\ xxx \\ xxx \\ xxx \\ xxx \\ xxx \\ xxx \\ xxx \\ xxx \\ xxx \\ xxx \\ xxx \\ xxx \\ xxx \\ xxx \\ xxx \\ xxx \\ xxx \\ xxx \\ xxx \\ xxx \\ xxx \\ xxx \\ xxx \\ xxx \\ xxx \\ xxx \\ xxx \\ xxx \\ xxx \\ xxx \\ xxx \\ xxx \\ xxx \\ xxx \\ xxx \\ xxx \\ xxx \\ xxx \\ xxx \\ xxx \\ xxx \\ xxx \\ xxx \\ xxx \\ xxx \\ xxx \\ xxx \\ xxx \\ xxx \\ xxx \\ xxx \\ xxx \\ xxx \\ xxx \\ xxx \\ xxx \\ xxx \\ xxx \\ xxx \\ xxx \\ xxx \\ xxx \\ xxx \\ xxx \\ xxx \\ xxx \\ xxx \\ xxx \\ xxx \\ xxx \\ xxx \\ xxx \\ xxx \\ xxx \\ 
\section{foooo}
xxx \\ xxx \\ xxx \\ xxx \\ xxx \\ xxx \\ xxx \\ xxx \\ xxx \\ xxx \\ xxx \\ xxx \\ xxx \\ xxx \\ xxx \\ xxx \\ xxx \\ xxx \\ xxx \\ xxx \\ xxx \\ xxx \\ xxx \\ xxx \\ xxx \\ xxx \\ xxx \\ xxx \\ xxx \\ xxx \\ xxx \\ xxx \\ xxx \\ xxx \\ xxx \\ xxx \\ xxx \\ xxx \\ xxx \\ xxx \\ xxx \\ xxx \\ xxx \\ xxx \\ xxx \\ xxx \\ xxx \\ xxx \\ xxx \\ xxx \\ xxx \\ xxx \\ xxx \\ xxx \\ xxx \\ xxx \\ xxx \\ xxx \\ xxx \\ xxx \\ xxx \\ xxx \\ xxx \\ xxx \\ xxx \\ xxx \\ xxx \\ xxx \\ xxx \\ xxx \\ xxx \\ xxx \\ xxx \\ xxx \\ xxx \\ xxx \\ xxx \\ xxx \\ xxx \\ xxx \\ 

\renewcommand{\thechapter}{}
\chapter{Implementierung}
\renewcommand{\thechapter}{3}
%
%%%%%%%%%%%%%%%%%%%%%%%%%%%%%%%%%%%%%%%%%%%%%%%%% *titel von section eventuell subsection wegen uebersicht*
%
\section{fooX}
xxx \\ xxx \\ xxx \\ xxx \\ xxx \\ xxx \\ xxx \\ xxx \\ xxx \\ xxx \\ xxx \\ xxx \\ xxx \\ xxx \\ xxx \\ xxx \\ xxx \\ xxx \\ xxx \\ xxx \\ xxx \\ xxx \\ xxx \\ xxx \\ xxx \\ xxx \\ xxx \\ xxx \\ xxx \\ xxx \\ xxx \\ xxx \\ xxx \\ xxx \\ xxx \\ xxx \\ xxx \\ xxx \\ xxx \\ xxx \\ xxx \\ xxx \\ xxx \\ xxx \\ xxx \\ xxx \\ xxx \\ xxx \\ xxx \\ xxx \\ xxx \\ xxx \\ xxx \\ xxx \\ xxx \\ xxx \\ xxx \\ xxx \\ xxx \\ xxx \\ xxx \\ xxx \\ xxx \\ xxx \\ xxx \\ xxx \\ xxx \\ xxx \\ xxx \\ xxx \\ xxx \\ xxx \\ xxx \\ xxx \\ xxx \\ xxx \\ xxx \\ xxx \\ xxx \\ xxx \\ 
\section{fooooX}
xxx \\ xxx \\ xxx \\ xxx \\ xxx \\ xxx \\ xxx \\ xxx \\ xxx \\ xxx \\ xxx \\ xxx \\ xxx \\ xxx \\ xxx \\ xxx \\ xxx \\ xxx \\ xxx \\ xxx \\ xxx \\ xxx \\ xxx \\ xxx \\ xxx \\ xxx \\ xxx \\ xxx \\ xxx \\ xxx \\ xxx \\ xxx \\ xxx \\ xxx \\ xxx \\ xxx \\ xxx \\ xxx \\ xxx \\ xxx \\ xxx \\ xxx \\ xxx \\ xxx \\ xxx \\ xxx \\ xxx \\ xxx \\ xxx \\ xxx \\ xxx \\ xxx \\ xxx \\ xxx \\ xxx \\ xxx \\ xxx \\ xxx \\ xxx \\ xxx \\ xxx \\ xxx \\ xxx \\ xxx \\ xxx \\ xxx \\ xxx \\ xxx \\ xxx \\ xxx \\ xxx \\ xxx \\ xxx \\ xxx \\ xxx \\ xxx \\ xxx \\ xxx \\ xxx \\ xxx \\ 
\end{document}